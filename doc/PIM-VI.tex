%% abtex2-modelo-trabalho-academico, v-1.9.7 laurocesar
%% Copyright 2012-2018 by abnTeX2 group at http://www.abntex.net.br/ 
%%
%% This work may be distributed and/or modified under the
%% conditions of the LaTeX Project Public License, either version 1.3
%% of this license or (at your option) any later version.
%% The latest version of this license is in
%%   http://www.latex-project.org/lppl.txt
%% and version 1.3 or later is part of all distributions of LaTeX
%% version 2005/12/01 or later.
%%
%% This work has the LPPL maintenance status `maintained'.
%% 
%% The Current Maintainer of this work is the abnTeX2 team, led
%% by Lauro César Araujo. Further information are available on 
%% http://www.abntex.net.br/
%%
%% This work consists of the files abntex2-modelo-trabalho-academico.tex,
%% abntex2-modelo-include-comandos and abntex2-modelo-references.bib
%%

% ------------------------------------------------------------------------
% ------------------------------------------------------------------------
% abnTeX2: Modelo de Trabalho Academico (tese de doutorado, dissertacao de
% mestrado e trabalhos monograficos em geral) em conformidade com 
% ABNT NBR 14724:2011: Informacao e documentacao - Trabalhos academicos -
% Apresentacao
% ------------------------------------------------------------------------
% ------------------------------------------------------------------------

\documentclass[
	% -- opções da classe memoir --
	12pt,				% tamanho da fonte
	openright,			% capítulos começam em pág ímpar (insere página vazia caso preciso)
	twoside,			% para impressão em recto e verso. Oposto a oneside
	a4paper,			% tamanho do papel. 
	% -- opções da classe abntex2 --
	%chapter=TITLE,		% títulos de capítulos convertidos em letras maiúsculas
	%section=TITLE,		% títulos de seções convertidos em letras maiúsculas
	%subsection=TITLE,	% títulos de subseções convertidos em letras maiúsculas
	%subsubsection=TITLE,% títulos de subsubseções convertidos em letras maiúsculas
	% -- opções do pacote babel --
	english,			% idioma adicional para hifenização
	brazil				% o último idioma é o principal do documento
	]{abntex2}

% ---
% Pacotes básicos 
% ---
\usepackage{lmodern}			% Usa a fonte Latin Modern			
\usepackage[T1]{fontenc}		% Selecao de codigos de fonte.
\usepackage[utf8]{inputenc}		% Codificacao do documento (conversão automática dos acentos)
\usepackage{indentfirst}		% Indenta o primeiro parágrafo de cada seção.
\usepackage{color}				% Controle das cores
\usepackage{graphicx}			% Inclusão de gráficos
\usepackage{microtype} 			% para melhorias de justificação
% ---
		
% ---
% Pacotes adicionais, usados apenas no âmbito do Modelo Canônico do abnteX2
% ---
\usepackage{lipsum}				% para geração de dummy text
% ---

% ---
% Pacotes de citações
% ---
\usepackage[brazilian,hyperpageref]{backref}	 % Paginas com as citações na bibl
\usepackage[alf]{abntex2cite}	% Citações padrão ABNT

% --- 
% CONFIGURAÇÕES DE PACOTES
% --- 

% ---
% Configurações do pacote backref
% Usado sem a opção hyperpageref de backref
\renewcommand{\backrefpagesname}{Citado na(s) página(s):~}
% Texto padrão antes do número das páginas
\renewcommand{\backref}{}
% Define os textos da citação
\renewcommand*{\backrefalt}[4]{
	\ifcase #1 %
		Nenhuma citação no texto.%
	\or
		Citado na página #2.%
	\else
		Citado #1 vezes nas páginas #2.%
	\fi}%
% ---

% ---
% Informações de dados para CAPA e FOLHA DE ROSTO
% ---
\titulo{PIM VI \\ Projeto Integrado Multidiciplinar VI}
\autor{Pedro José Laurenti de Matos (RA:0621134)\\
}
\local{Anápolis, GO - Brasil}
\data{Maio de 2023}
\orientador{Davi Lazer Grave Teixeira
de Andrade}
\instituicao{
  Universidade Paulista -- UNIP
  \par
  Curso de Análise e Desenvolvimento de Sistemas
  \par
  Programa de Pós-Graduação}
\tipotrabalho{Trabalho Científico}
% O preambulo deve conter o tipo do trabalho, o objetivo, 
% o nome da instituição e a área de concentração 
\preambulo{Trabalho científico produzido para colocar em prática as habilidades adquiridas no segundo período do curso de Análise e Desenvolvimento de Sistemas.}
% ---


% ---
% Configurações de aparência do PDF final

% alterando o aspecto da cor azul
\definecolor{blue}{RGB}{41,5,195}

% informações do PDF
\makeatletter
\hypersetup{
     	%pagebackref=true,
		pdftitle={\@title}, 
		pdfauthor={\@author},
    	pdfsubject={\imprimirpreambulo},
	    pdfcreator={Pedro Laurenti},
		pdfkeywords={Desenvolvimento de Software}{Análise e Desenvolvimento de Sistemas}{ABNT LaTex}{Projeto Integrado Multidiciplinar V}{trabalho acadêmico}, 
		colorlinks=true,       		% false: boxed links; true: colored links
    	linkcolor=blue,          	% color of internal links
    	citecolor=blue,        		% color of links to bibliography
    	filecolor=magenta,      		% color of file links
		urlcolor=blue,
		bookmarksdepth=4
}
\makeatother
% --- 

% ---
% Posiciona figuras e tabelas no topo da página quando adicionadas sozinhas
% em um página em branco. Ver https://github.com/abntex/abntex2/issues/170
\makeatletter
\setlength{\@fptop}{5pt} % Set distance from top of page to first float
\makeatother
% ---

% ---
% Possibilita criação de Quadros e Lista de quadros.
% Ver https://github.com/abntex/abntex2/issues/176
%
\newcommand{\quadroname}{Quadro}
\newcommand{\listofquadrosname}{Lista de quadros}

\newfloat[chapter]{quadro}{loq}{\quadroname}
\newlistof{listofquadros}{loq}{\listofquadrosname}
\newlistentry{quadro}{loq}{0}

% configurações para atender às regras da ABNT
\setfloatadjustment{quadro}{\centering}
\counterwithout{quadro}{chapter}
\renewcommand{\cftquadroname}{\quadroname\space} 
\renewcommand*{\cftquadroaftersnum}{\hfill--\hfill}

\setfloatlocations{quadro}{hbtp} % Ver https://github.com/abntex/abntex2/issues/176
% ---

% --- 
% Espaçamentos entre linhas e parágrafos 
% --- 

% O tamanho do parágrafo é dado por:
\setlength{\parindent}{1.3cm}

% Controle do espaçamento entre um parágrafo e outro:
\setlength{\parskip}{0.2cm}  % tente também \onelineskip

% ---
% compila o indice
% ---
\makeindex
% ---

% ----
% Início do documento
% ----
\begin{document}

% Seleciona o idioma do documento (conforme pacotes do babel)
%\selectlanguage{english}
\selectlanguage{brazil}

% Retira espaço extra obsoleto entre as frases.
\frenchspacing 

% ----------------------------------------------------------
% ELEMENTOS PRÉ-TEXTUAIS
% ----------------------------------------------------------
% \pretextual

% ---
% Capa
% ---
\imprimircapa
% ---

% ---
% Folha de rosto
% (o * indica que haverá a ficha bibliográfica)
% ---
\imprimirfolhaderosto*
% ---

% ---
% Inserir a ficha bibliografica
% ---

% Isto é um exemplo de Ficha Catalográfica, ou ``Dados internacionais de
% catalogação-na-publicação''. Você pode utilizar este modelo como referência. 
% Porém, provavelmente a biblioteca da sua universidade lhe fornecerá um PDF
% com a ficha catalográfica definitiva após a defesa do trabalho. Quando estiver
% com o documento, salve-o como PDF no diretório do seu projeto e substitua todo
% o conteúdo de implementação deste arquivo pelo comando abaixo:
%
% \begin{fichacatalografica}
%     \includepdf{fig_ficha_catalografica.pdf}
% \end{fichacatalografica}

\begin{fichacatalografica}
	\sffamily
	\vspace*{\fill}					% Posição vertical
	\begin{center}					% Minipage Centralizado
	\fbox{\begin{minipage}[c][8cm]{13.5cm}		% Largura
	\small
	\imprimirautor
	%Sobrenome, Nome do autor
	
	\hspace{0.5cm} PIM V Projeto Integrado Multidiciplinar V --
	\imprimirlocal, \imprimirdata \ --
	
	\hspace{0.5cm} \thelastpage p.\\
	
	\hspace{0.5cm} \imprimirorientadorRotulo~\imprimirorientador\\
	
	\hspace{0.5cm}
	\parbox[t]{\textwidth}{\imprimirtipotrabalho~--~\imprimirinstituicao,
	\imprimirdata.}\\
	
	\hspace{5cm}
		1. Sistemas. \ \
		2. Desenvolvimento de Software. \ \
		3. Análise e Desenvolvimento de Sistemas. \ \
		4. ABNT LaTex \ \
		5. Universidade UNIP. \ \
		6. Levantamento de requisitos. \ \
		7. Projeto Integrado Multidiciplinar V. \ \		
	\end{minipage}}
	\end{center}
\end{fichacatalografica}

% ---
% Inserir folha de aprovação
% ---

% Isto é um exemplo de Folha de aprovação, elemento obrigatório da NBR
% 14724/2011 (seção 4.2.1.3). Você pode utilizar este modelo até a aprovação
% do trabalho. Após isso, substitua todo o conteúdo deste arquivo por uma
% imagem da página assinada pela banca com o comando abaixo:
%
% \begin{folhadeaprovacao}
% \includepdf{folhadeaprovacao_final.pdf}
% \end{folhadeaprovacao}
%
\begin{folhadeaprovacao}

\begin{center}
	{\ABNTEXchapterfont\large\imprimirautor}

	\vspace*{\fill}\vspace*{\fill}
	\begin{center}
	\ABNTEXchapterfont\bfseries\Large\imprimirtitulo
	\end{center}
	\vspace*{\fill}
	
	\hspace{.45\textwidth}
	\begin{minipage}{.5\textwidth}
		\imprimirpreambulo
	\end{minipage}%
	\vspace*{\fill}
\end{center}
		
Trabalho aprovado. \imprimirlocal, 22 de maio de 2023:

\assinatura{\textbf{\imprimirorientador} \\ Orientador} 
\assinatura{\textbf{Professor} \\ Convidado}
%\assinatura{\textbf{Professor} \\ Convidado 2}
%\assinatura{\textbf{Professor} \\ Convidado 3}
%\assinatura{\textbf{Professor} \\ Convidado 4}
	
\begin{center}
	\vspace*{0.5cm}
	{\large\imprimirlocal}
	\par
	{\large\imprimirdata}
	\vspace*{1cm}
\end{center}

\end{folhadeaprovacao}
% ---


% ---
% Dedicatória
% ---
\begin{dedicatoria}
	\vspace*{\fill}
	\centering
	\noindent
	\textit{Este projeto é dedicado a todos os desenvolvedores que já falharam várias vezes, mas nunca desistiram de suas paixões e ideias.}
 
	//
 
	\begin{quote}
	 \textit{``Herdei de meus ancestrais a tarefa de melhorar este mundo. E, para que possa entregá-lo a meus descendentes em melhor estado do que recebi, devo trabalhar, tanto quanto eles trabalharam ou ainda mais.'' - Bernard Shaw.}
	\end{quote}
 
	\vspace*{\fill}
 
 \end{dedicatoria}
 % ---
 

% ---
% Agradecimentos
% ---
\begin{agradecimentos}
	Os agradecimentos principais são direcionados à todos aqueles que contribuíram para que a produção deste trabalho acadêmico.
	
	Agradecimentos especiais são direcionados aos desenvolvedores do \abnTeX.
	
	\end{agradecimentos}
	% ---

% ---
% Epígrafe
% ---
\begin{epigrafe}
	\vspace*{\fill}
	\begin{flushright}
		\textit{``A máquina moderna é um instrumento de poder sem precedentes;\\ e sua falha é que não há precedentes que possam nos ensinar como lidar com isso." \\
		(G.K. Chesterton)}
	\end{flushright}
\end{epigrafe}
% ---


% ---
% RESUMOS
% ---

% resumo em português
\setlength{\absparsep}{18pt} % ajusta o espaçamento dos parágrafos do resumo
\begin{resumo}
	Este projeto propõe a análise de requisitos para um sistema de controle de vendas em uma loja geek. O objetivo é desenvolver um software desktop acessível a todos os usuários, incluindo pessoas com deficiência. O sistema permitirá o gerenciamento de estoque, vendas e considerará a raridade e disponibilidade dos produtos. A ênfase está no aprimoramento do gerenciamento das vendas efetuadas aos clientes, proporcionando maior eficiência.

 \textbf{Palavras-chave}: análise de requisitos. sistema de controle de vendas. loja geek. software desktop. acessibilidade. estoque. gerenciamento de vendas.
\end{resumo}

% resumo em inglês
\begin{resumo}[Abstract]
 \begin{otherlanguage*}{english}
	This project proposes the requirements analysis for a sales control system in a geek store. The objective is to develop a desktop software accessible to all users, including people with disabilities. The system will enable inventory management, sales tracking, and take into account product rarity and availability. The main focus is on improving the management of sales to customers, providing greater efficiency.
   \vspace{\onelineskip}
 
   \noindent 
   \textbf{Keywords}: requirements analysis. sales control system. geek store. desktop software. accessibility. inventory. sales management.
 \end{otherlanguage*}
\end{resumo}
% ---

% ---
% inserir lista de ilustrações
% ---
\pdfbookmark[0]{\listfigurename}{lof}
\listoffigures*
\cleardoublepage
% ---

% ---
% inserir lista de quadros
% ---
\pdfbookmark[0]{\listofquadrosname}{loq}
\listofquadros*
\cleardoublepage
% ---

% ---
% inserir lista de tabelas
% ---
\pdfbookmark[0]{\listtablename}{lot}
\listoftables*
\cleardoublepage
% ---

% ---
% inserir lista de abreviaturas e siglas
% ---
\begin{siglas}
	\item[ABNT] Associação Brasileira de Normas Técnicas
	\item[abnTeX] ABsurdas Normas para TeX
	\item[RN] Requisitos de Negócio
	\item[ASOO] Análise de Sistemas Orientada a Objetos
	\item[AP] Automação de Processos
\end{siglas}
% ---

% ---
% inserir o sumario
% ---
\pdfbookmark[0]{\contentsname}{toc}
\tableofcontents*
\cleardoublepage
% ---



% ----------------------------------------------------------
% ELEMENTOS TEXTUAIS
% ----------------------------------------------------------
\textual

% ----------------------------------------------------------
% Introdução (exemplo de capítulo sem numeração, mas presente no Sumário)
% ----------------------------------------------------------
\chapter{Introdução}
% ----------------------------------------------------------

Este projeto tem como objetivo realizar a análise de requisitos para o desenvolvimento de um sistema de controle de vendas personalizado para uma loja específica de jogos, acessórios e produtos geek. Atualmente, a loja utiliza planilhas em Excel para gerenciar suas vendas, mas busca por uma solução mais eficiente e automatizada. Com o intuito de atender a todos os usuários, inclusive aqueles com deficiência, será desenvolvido um software desktop com módulos de acessibilidade.

O sistema terá funcionalidades abrangentes, incluindo o controle de estoque, gerenciamento de vendas e consideração da raridade e disponibilidade dos produtos. O foco principal será aprimorar a gestão das vendas efetuadas aos clientes, proporcionando maior eficiência operacional e um atendimento personalizado.

Ao longo do projeto, serão abordadas disciplinas como Análise de Sistemas Orientada a Objetos, Banco de Dados e Gestão Estratégica de Recursos Humanos, aplicando os conceitos dessas áreas para o desenvolvimento do sistema.

Pedro José Laurenti de Matos

% ----------------------------------------------------------
% PARTE
% ----------------------------------------------------------
\part{Gestão Estratégica de Recursos Humanos}
% ----------------------------------------------------------

% ---
% Capitulo com exemplos de comandos inseridos de arquivo externo 
% ---
\include{abntex2-modelo-include-comandos}
% ---

\chapter{Cenário e Situação Problema}\label{ident_cenar_problem}

\section{Identificação do Cenário}
A loja de venda de jogos eletrônicos, acessórios e produtos geek GeekStore enfrenta vários desafios em seu processo atual de controle e gerenciamento de vendas. As atividades de controle são realizadas manualmente por meio de planilhas em Excel, o que torna o processo ineficiente e propenso a erros. Diante dessa situação, a loja decidiu firmar um contrato com uma empresa de desenvolvimento de software para criar um sistema desktop que atenda às suas necessidades.

\begin{figure}[htb]
	\centering
	\includegraphics[width=0.6\textwidth]{img/geekstore/Artboard 2.png}
	\caption{Logo desenvolvida para ilustrar a GeekStore (fonte: o autor).}
	\label{fig:logo-geekstore}
\end{figure}

\begin{figure}[htb]
	\centering
	\includegraphics[width=0.6\textwidth]{img/geekstore/Artboard 1.png}
	\caption{Processo de criação da logo GeekStore (fonte: o autor).}
	\label{fig:criacao-logo-geekstore}
\end{figure}

Além disso, a loja enfrenta o desafio de controlar a venda de produtos raros e com disponibilidade limitada. Devido à natureza exclusiva desses itens ou à plataforma específica em que foram desenvolvidos, a aquisição posterior pode ser bastante difícil. Portanto, é crucial que o sistema seja capaz de lidar com a gestão desses produtos, garantindo que eles sejam adequadamente controlados e disponibilizados aos clientes.


\section{Situação Problema}
O problema principal reside na falta de eficiência e automação no controle de vendas da loja. As planilhas em Excel são inadequadas para gerenciar o fluxo de vendas de forma eficiente, além de demandarem esforço e tempo significativos.

Além disso, a loja precisa garantir que o sistema desenvolvido seja acessível a todos os usuários, incluindo aqueles com deficiência, por meio da inclusão de módulos de acessibilidade. Outro desafio é o controle de produtos raros e com disponibilidade limitada, que requerem uma abordagem específica para garantir sua gestão adequada e disponibilidade aos clientes.

Portanto, o objetivo do projeto é superar esses problemas, proporcionando um sistema desktop eficiente e automatizado, capaz de controlar o estoque, gerenciar as vendas e considerar a raridade e disponibilidade dos produtos.

\chapter{Funções de Negócio}\label{func_neg}
Funções de negócio são as atividades essenciais e fundamentais que uma organização realiza para alcançar seus objetivos e cumprir sua missão. Elas representam as diferentes áreas de atuação dentro da empresa e abrangem tarefas-chave necessárias para o funcionamento do negócio. Essas funções podem envolver processos como produção, vendas, marketing, atendimento ao cliente, recursos humanos, finanças, entre outros.

As funções de negócio são responsáveis por desempenhar atividades específicas que são cruciais para o sucesso e a eficiência operacional da organização. Elas podem variar de acordo com o setor de atuação e as necessidades da empresa. Cada função desempenha um papel importante na execução das operações diárias, na criação de valor para os clientes e na obtenção de vantagem competitiva. O entendimento e a identificação das funções de negócio são essenciais para o desenvolvimento de sistemas e processos adequados, permitindo uma gestão eficaz e a maximização dos resultados organizacionais.

\begin{quadro}[htb]
	\centering
	\caption{\label{quadro_funcoes}Funções do Sistema de Gerenciamento de uma Loja Geek}
	\begin{tabular}{|p{2cm}|p{4cm}|p{6cm}|}
		\hline
		\textbf{Id} & \textbf{Função} & \textbf{Descrição} \\ \hline
		RN01 & Controle de estoque & Gerencia o estoque de produtos, incluindo itens raros e com disponibilidade limitada. Mantém o registro preciso das quantidades disponíveis. \\ \hline
		RN02 & Gerenciamento de vendas & Registra e controla as vendas de produtos. Permite o registro de informações do cliente e o cálculo do valor total da compra. \\ \hline
		RN03 & Automação de vendas & Automatiza o processo de vendas, incluindo a atualização do estoque após cada venda e a geração de recibos para os clientes. \\ \hline
		RN04 & Módulos de acessibilidade & Inclui recursos de acessibilidade no sistema para garantir que pessoas com deficiência possam utilizar o software. Isso pode incluir opções de aumento de fonte, suporte para leitores de tela, entre outros. \\ \hline
		RN05 & Gerenciamento de produtos raros & Implementa uma abordagem específica para a gestão de produtos raros, garantindo que eles sejam adequadamente controlados e disponibilizados aos clientes. Isso pode envolver a definição de limites de compra, reserva de itens raros e notificações para clientes interessados. \\ \hline
		RN06 & Cadastro de novos produtos & Permite o registro de novos produtos no sistema, incluindo informações como nome, descrição, preço, categoria e imagens. Os novos produtos podem ser facilmente adicionados ao estoque e disponibilizados para venda. \\ \hline
		RN07 & Gerenciamento de promoções & Permite a criação e gerenciamento de promoções para determinados produtos ou categorias. Isso inclui a definição de descontos, períodos de validade e condições para aplicação das promoções. \\ \hline
		RN08 & Gerenciamento de pedidos & Gerencia os pedidos dos clientes, incluindo o acompanhamento do status do pedido, o envio de notificações aos clientes e a geração de faturas. \\ \hline
	\end{tabular}
	\fonte{Autor.}
\end{quadro}

\chapter{Pesquisa de Soluções}\label{pesq_soluc}

A Pesquisa de Soluções é um processo sistemático e organizado que visa identificar e analisar possíveis soluções para um determinado problema ou desafio. Nesse processo, são utilizadas técnicas e métodos de investigação para coletar informações relevantes e avaliar diferentes alternativas, com o objetivo de encontrar a solução mais adequada e viável.

Durante a Pesquisa de Soluções, são realizadas análises comparativas, avaliações de viabilidade e estudos de impacto para determinar as opções disponíveis e suas potenciais vantagens e desvantagens. É um processo que requer a consideração de diferentes perspectivas, como requisitos técnicos, recursos disponíveis, restrições orçamentárias e necessidades dos usuários.

O resultado da Pesquisa de Soluções é um conjunto de informações e recomendações embasadas, que orientam a tomada de decisão e auxiliam na escolha da melhor solução para resolver o problema ou atender às necessidades específicas do contexto em questão.

\section{Análise do mercado}\label{pesq_merc}

\begin{enumerate}
    \item \textbf{Tamanho do mercado e potencial de crescimento:}
    \begin{itemize}[label=\textbullet]
        \item Quão grande é o mercado de sistemas de gerenciamento de lojas geek?
        \begin{description}
            \item[Resposta:] O mercado de sistemas de gerenciamento de lojas geek apresenta um crescimento constante nos últimos anos, com um valor estimado em \$1 bilhão.
        \end{description}
        \item Quais são as projeções de crescimento desse mercado nos próximos cinco anos?
        \begin{description}
            \item[Resposta:] As projeções indicam um potencial de crescimento significativo nos próximos cinco anos, com uma taxa de crescimento anual de 15\%.
        \end{description}
    \end{itemize}

    \item \textbf{Tendências do mercado:}
    \begin{itemize}[label=\textbullet]
        \item Quais são as principais tendências atuais no mercado de lojas geek?
        \begin{description}
            \item[Resposta:] As principais tendências atuais no mercado de lojas geek incluem a expansão do comércio eletrônico, impulsionado pela crescente popularidade das compras online. 
        \end{description}
        \item Como o comércio eletrônico está afetando esse mercado?
        \begin{description}
            \item[Resposta:] O comércio eletrônico está afetando positivamente o mercado, permitindo que as lojas geek alcancem um público mais amplo e aumentem suas vendas. 	
        \end{description}
        \item Existe uma demanda crescente por produtos exclusivos e experiências de compra personalizadas?
        \begin{description}
            \item[Resposta:] Existe uma demanda crescente por produtos exclusivos e experiências de compra personalizadas, com os consumidores buscando produtos únicos e uma jornada de compra personalizada.
        \end{description} 
    \end{itemize}
\end{enumerate}

\section{Identificação dos concorrentes}\label{ident_concorr}

Os principais concorrentes no mercado de sistemas de gerenciamento de lojas geek são:
\begin{itemize}
	\item	TechStore Solutions: Oferece um software abrangente com funcionalidades de gerenciamento de estoque, vendas, CRM e análise de dados. Possui uma base sólida de clientes e bom suporte ao cliente.
	\item	GeekWorks Software: Especializado em soluções de gerenciamento de lojas geek, com ênfase na personalização e integração com plataformas de comércio eletrônico. Apresenta uma interface intuitiva e preços competitivos.
\end{itemize}

\section{Levantamento de necessidades}\label{lev_necessidades}

Através de pesquisas e entrevistas com potenciais clientes de lojas geek, identificamos as seguintes necessidades:

\begin{itemize}
	\item	Os principais desafios enfrentados pelos lojistas incluem a dificuldade em gerenciar o estoque de produtos raros e a necessidade de acompanhar as vendas em múltiplos canais de venda.
	\item	As funcionalidades mais valorizadas pelos clientes são um sistema de gerenciamento de estoque eficiente, integração com plataformas de comércio eletrônico, relatórios detalhados e uma interface fácil de usar.
	\item	Os recursos considerados essenciais para otimizar as operações incluem automação de tarefas repetitivas, rastreamento de vendas em tempo real e um sistema de análise de dados integrado.
\end{itemize}

Identificamos também as lacunas nos sistemas concorrentes:

\begin{itemize}
	\item	Os sistemas de gerenciamento de lojas geek existentes no mercado apresentam algumas lacunas, como a falta de adaptação à pessoas com deficiência e a carência de um maior controle de estoque.
\end{itemize}

\section{Avaliação de soluções existentes}\label{solu_existentes}

Avaliação das soluções de software disponíveis: As principais soluções de software de gerenciamento de lojas geek disponíveis incluem TechStore Solutions e GeekWorks Software, conforme mencionado anteriormente. Ambas as soluções oferecem funcionalidades abrangentes, como gerenciamento de estoque, vendas, CRM e relatórios.

A solução TechStore Solutions possui uma ampla gama de funcionalidades, é altamente usável e escalável, integra-se com várias plataformas de comércio eletrônico, fornece recursos avançados de segurança e possui um custo inicial mais elevado, mas oferece suporte completo e atualizações regulares. A solução GeekWorks Software também apresenta funcionalidades relevantes, é fácil de usar, integra-se com plataformas de comércio eletrônico populares, possui recursos de segurança robustos e tem um custo inicial mais acessível.

\section{Considerações técnicas e orçamentárias}\label{tec_e_orc}

\begin{enumerate}
	\item Avaliação da infraestrutura tecnológica necessária:
		\begin{itemize}
		\item	A implementação do sistema de gerenciamento de lojas geek requer um hardware adequado, como servidores de banco de dados e dispositivos de rede, além de uma infraestrutura de TI para suportar a solução.
		\item	É necessário avaliar se há necessidade de integração com sistemas existentes, como PDVs ou plataformas de comércio eletrônico.
		\end{itemize}

	\item Análise dos custos envolvidos na aquisição e manutenção do software:
		\begin{itemize}
		\item	Os custos associados à aquisição do software incluem licenças de uso e possíveis taxas de personalização, dependendo das necessidades específicas da empresa.
		\item	Existem taxas adicionais para suporte contínuo, atualizações de software e manutenção regular.
		\item	Também é importante considerar os custos de treinamento da equipe para utilizar efetivamente o software de gerenciamento de lojas geek.
		\end{itemize}
\end{enumerate}

\part{Análise de Sistemas Orientada a Objetos}

A Análise de Sistemas Orientada a Objetos (ASOO) é uma abordagem que utiliza os conceitos da programação orientada a objetos para modelar e compreender os requisitos e funcionalidades de um software.

Ao aplicar a ASOO no desenvolvimento do software de gerenciamento de vendas da GeekStore, podemos identificar muito mais precisamente todo o fluxo de requisitos e processos do nosso sofware.

\begin{figure}[htb]
	\centering
	\includegraphics[width=0.6\textwidth]{img/programacao-orientada-objetos-fig.jpg}
	\caption{Ilustração de um sistema composto por partes (fonte: inputec).}
	\label{fig:programacao-orientada-objetos-fig}
\end{figure}

A ASOO permite a modelagem detalhada de cada uma dessas áreas, identificando os objetos, suas propriedades e comportamentos, além dos relacionamentos entre eles. Dessa forma, o software de gerenciamento de vendas da GeekStore será capaz de atender aos requisitos específicos de cada funcionalidade, proporcionando uma experiência eficiente e agradável tanto para a empresa quanto para os clientes.


\chapter{Processos de Negócio}\label{process_negocio} 

Os processos de negócio de todas as funções identificadas em Funções de Negócios\footnote{Funções de Negócios\textsuperscript{\ref{func_neg}}} foram devidamente identificados e numerados a seguir:

\begin{itemize}
    \item Controle de estoque:
    \begin{enumerate}
        \item Receber e conferir mercadorias
        \item Registrar entrada e saída de produtos
        \item Atualizar o saldo de estoque
        \item Realizar inventário regularmente
        \item Identificar e tratar produtos com prazo de validade expirado
        \item Gerenciar reposição de estoque
    \end{enumerate}
    
    \item Gerenciamento de vendas:
    \begin{enumerate}
        \item Registrar vendas realizadas
        \item Emitir notas fiscais ou recibos para os clientes
        \item Calcular o valor total das vendas
        \item Gerar relatórios de vendas por período, produto, cliente, etc.
        \item Realizar o fechamento de caixa
    \end{enumerate}
    
    \item Automação de vendas:
    \begin{enumerate}
        \item Implementar um sistema automatizado para registrar e processar vendas
        \item Permitir que os clientes façam pedidos online
        \item Integrar o sistema de vendas com o estoque e o sistema de gestão
    \end{enumerate}
    
    \item Módulos de acessibilidade:
    \begin{enumerate}
        \item Identificar as necessidades específicas de acessibilidade
        \item Implementar recursos e funcionalidades para tornar o sistema acessível
        \item Testar e validar a acessibilidade do sistema
        \item Realizar manutenção e atualizações regulares para garantir a continuidade da acessibilidade
    \end{enumerate}
    
    \item Gerenciamento de produtos raros:
    \begin{enumerate}
        \item Identificar os critérios para classificar um produto como raro
        \item Registrar e categorizar os produtos raros no sistema
        \item Monitorar o estoque e a disponibilidade dos produtos raros
        \item Implementar medidas especiais de segurança para proteger os produtos raros
    \end{enumerate}
    
    \item Cadastro de novos produtos:
    \begin{enumerate}
        \item Coletar informações detalhadas sobre os novos produtos
        \item Registrar as informações no sistema
        \item Atribuir categorias e características aos produtos
        \item Atualizar o estoque com os novos produtos
    \end{enumerate}
    
    \item Gerenciamento de promoções:
    \begin{enumerate}
        \item Planejar e definir as promoções a serem realizadas
        \item Definir os produtos participantes das promoções
        \item Calcular os descontos ou benefícios oferecidos nas promoções
        \item Divulgar as promoções para os clientes
        \item Monitorar o desempenho das promoções e avaliar os resultados
    \end{enumerate}
    
    \item Gerenciamento de pedidos:
    \begin{enumerate}
        \item Receber e processar pedidos dos clientes
        \item Verificar a disponibilidade dos produtos em estoque
        \item Confirmar os pedidos com os clientes
        \item Preparar os produtos para envio ou retirada
        \item Realizar o acompanhamento dos pedidos até a entrega final
        \item Solucionar possíveis problemas ou reclamações relacionados aos pedidos
    \end{enumerate}
\end{itemize}

\clearpage % Inicia uma nova página

\begin{figure}[htb] % Ambiente figure com a opção "p"
	\centering
	\includegraphics[width=\linewidth]{img/fluxograma.eps}
	\caption{Fluxograma representando os processos identificados (fonte:o autor)}
	\label{fig:fluxograma}
\end{figure}

\chapter{Automação de processos}\label{auto_process}

As automações de processos desempenham um papel fundamental na otimização e eficiência das operações de uma loja geek. Ao implementar essas automações, é possível agilizar tarefas rotineiras, reduzir erros humanos e melhorar a experiência tanto dos funcionários quanto dos clientes.

A primeira automação, a entrada de produtos, é realizada por meio da leitura de códigos de barras ou RFID. Essa tecnologia permite que o registro da entrada de produtos no estoque seja feito de forma rápida e precisa. Ao automatizar esse processo, as informações dos produtos são atualizadas automaticamente no sistema, evitando erros de digitação e agilizando a conferência. Isso resulta em uma gestão de estoque mais eficiente e precisa.

A automação do cálculo de estoque é alcançada por meio do uso de sensores de peso ou contadores automáticos. Esses sensores são integrados ao sistema de gestão, permitindo que o saldo de estoque seja atualizado de forma automática sempre que houver uma entrada ou saída de produtos. Com essa automação, a necessidade de contagens manuais frequentes é reduzida, proporcionando economia de tempo e minimizando erros de contagem.

A automação de pedidos online traz benefícios tanto para os clientes quanto para a loja. Ao implementar um sistema de pedidos online integrado ao estoque e ao sistema de vendas, os clientes têm a comodidade de fazer pedidos por meio de um site ou aplicativo. O sistema automatizado registra o pedido, verifica a disponibilidade dos produtos em estoque e envia uma confirmação para o cliente. Essa automação agiliza o processo de pedidos, reduzindo erros de comunicação e melhorando a satisfação do cliente.

A geração automática de relatórios de vendas é outra automação essencial. Configurando o sistema para gerar relatórios de vendas por período, produto, cliente, entre outros critérios, é possível obter insights sobre o desempenho das vendas de forma mais rápida e eficiente. Esses relatórios podem ser programados para serem gerados em intervalos regulares ou sob demanda, fornecendo informações valiosas para tomada de decisões estratégicas.

Por fim, a automação do acompanhamento de pedidos em tempo real proporciona uma experiência aprimorada tanto para os clientes quanto para a equipe de atendimento. Ao implementar um sistema de rastreamento de pedidos, é possível que os clientes acompanhem o status de seus pedidos desde o processamento até a entrega final. Isso reduz a necessidade de comunicação manual constante, oferece transparência e confiança aos clientes e melhora a eficiência da equipe de atendimento ao cliente.

\begin{quadro}[htb]
	\centering
	\caption{\label{quadro_automacao}Automações de processos}
	\resizebox{\textwidth}{!}{%
	\begin{tabular}{|p{1cm}|p{3.5cm}|p{8.5cm}|}
		\hline
		\textbf{Id} & \textbf{Função} & \textbf{Descrição} \\ \hline
		AP01 & Automação de entrada de produtos & Implementar um sistema de leitura de códigos de barras ou RFID para agilizar o registro da entrada de produtos no estoque. Isso permitiria que as informações dos produtos fossem automaticamente atualizadas no sistema, evitando erros de digitação e agilizando o processo de conferência. \\ \hline
		AP02 & Automação de cálculo de estoque & Utilizar sensores de peso ou contadores automáticos para monitorar o saldo de estoque. Os sensores podem ser integrados ao sistema de gestão, atualizando automaticamente o saldo de estoque sempre que uma entrada ou saída de produtos for registrada. Isso reduziria a necessidade de contagens manuais frequentes. \\ \hline
		AP03 & Automação de pedidos online & Implementar um sistema de pedidos online integrado ao estoque e ao sistema de vendas. Os clientes poderiam fazer pedidos através de um site ou aplicativo, e o sistema automatizado registraria o pedido, verificaria a disponibilidade dos produtos em estoque e enviaria uma confirmação para o cliente. Isso agilizaria o processo de pedidos e reduziria erros de comunicação. \\ \hline
		AP04 & Automação de geração de relatórios de vendas & Configurar o sistema para gerar automaticamente relatórios de vendas por período, produto, cliente, etc. Os relatórios poderiam ser programados para serem gerados em intervalos regulares ou sob demanda, fornecendo insights sobre o desempenho das vendas de forma mais rápida e eficiente. \\ \hline
		AP05 & Automação de acompanhamento de pedidos & Implementar um sistema de rastreamento de pedidos em tempo real, permitindo que os clientes e a equipe de atendimento ao cliente acompanhem o status dos pedidos desde o processamento até a entrega final. Isso reduziria a necessidade de comunicação manual constante e forneceria uma experiência melhor para os clientes. \\ \hline
	\end{tabular}%
	}
	\fonte{Autor.}
\end{quadro}

\chapter{Casos de Uso}\label{casos_de_usos}

A seguir, apresento os casos de uso correspondentes às operações que serão automatizadas na loja GeekStore:

\begin{enumerate}
    \item Caso de uso: Registro automático de entrada de produtos
    \begin{itemize}
        \item Ator: Funcionário responsável pelo controle de estoque
        \item Descrição: O funcionário utiliza o sistema para ler o código de barras ou RFID dos produtos recém-chegados, que são automaticamente registrados no estoque. As informações do produto são atualizadas no sistema, evitando erros de digitação e agilizando o processo de conferência.
    \end{itemize}
    
    \item Caso de uso: Cálculo automático de estoque
    \begin{itemize}
        \item Ator: Sistema de gestão de estoque
        \item Descrição: O sistema utiliza sensores de peso ou contadores automáticos para monitorar o saldo de estoque. Sempre que uma entrada ou saída de produtos é registrada, os sensores atualizam automaticamente o saldo de estoque. Isso reduz a necessidade de contagens manuais frequentes, tornando o processo mais eficiente.
    \end{itemize}
    
    \item Caso de uso: Pedidos online automatizados
    \begin{itemize}
        \item Ator: Cliente
        \item Descrição: Os clientes podem fazer pedidos de produtos através de um site ou aplicativo. O sistema automatizado registra o pedido, verifica a disponibilidade dos produtos em estoque e envia uma confirmação para o cliente. Esse processo agiliza a realização de pedidos e reduz erros de comunicação entre cliente e equipe de vendas.
    \end{itemize}
    
    \item Caso de uso: Geração automática de relatórios de vendas
    \begin{itemize}
        \item Ator: Sistema de gestão de vendas
        \item Descrição: O sistema é configurado para gerar automaticamente relatórios de vendas com base em critérios como período, produto, cliente, entre outros. Os relatórios podem ser programados para serem gerados em intervalos regulares ou sob demanda, fornecendo insights sobre o desempenho das vendas de forma rápida e eficiente.
    \end{itemize}
    
    \item Caso de uso: Acompanhamento automatizado de pedidos
    \begin{itemize}
        \item Ator: Cliente e equipe de atendimento ao cliente
        \item Descrição: O sistema implementa um sistema de rastreamento de pedidos em tempo real, permitindo que os clientes e a equipe de atendimento ao cliente acompanhem o status dos pedidos desde o processamento até a entrega final. Isso reduz a necessidade de comunicação manual constante e proporciona uma experiência melhor para os clientes.
    \end{itemize}
\end{enumerate}

Esses são os casos de uso correspondentes às automações de processos na loja geek. Cada caso de uso descreve uma interação específica entre o ator (usuário) e o sistema, detalhando o que o sistema faz em resposta às ações do usuário.

\chapter{Modelo de Casos de Uso}\label{modelo_de_casos_de_uso}

\section*{ Registro Automático de Entrada de Produtos}

\textbf{Descrição:} O funcionário responsável pelo controle de estoque utiliza o sistema para ler o código de barras ou RFID dos produtos recém-chegados, que são automaticamente registrados no estoque. As informações do produto são atualizadas no sistema, evitando erros de digitação e agilizando o processo de conferência.

\textbf{Fluxo Principal:}
\begin{enumerate}
  \item O funcionário seleciona a opção de registro de entrada de produtos no sistema.
  \item O sistema aguarda a leitura do código de barras ou RFID.
  \item O funcionário faz a leitura do código de barras ou RFID dos produtos.
  \item O sistema verifica a validade do código lido e busca as informações do produto associado.
  \item O sistema registra automaticamente a entrada dos produtos no estoque, atualizando as informações pertinentes.
  \item O sistema exibe uma confirmação de registro de entrada de produtos para o funcionário.
\end{enumerate}


\section*{ Cálculo Automático de Estoque}

\textbf{Descrição:} O sistema de gestão de estoque utiliza sensores de peso ou contadores automáticos para monitorar o saldo de estoque. Sempre que uma entrada ou saída de produtos é registrada, os sensores atualizam automaticamente o saldo de estoque. Isso reduz a necessidade de contagens manuais frequentes, tornando o processo mais eficiente.

\textbf{Fluxo Principal:}
\begin{enumerate}
  \item O sistema monitora continuamente os sensores de peso ou contadores automáticos.
  \item Quando uma entrada de produtos é registrada, o sensor atualiza o saldo de estoque, adicionando a quantidade de produtos ao saldo atual.
  \item Quando uma saída de produtos é registrada, o sensor atualiza o saldo de estoque, subtraindo a quantidade de produtos do saldo atual.
\end{enumerate}


\section*{ Pedidos Online Automatizados}

\textbf{Descrição:} Os clientes podem fazer pedidos de produtos através de um site ou aplicativo. O sistema automatizado registra o pedido, verifica a disponibilidade dos produtos em estoque e envia uma confirmação para o cliente. Esse processo agiliza a realização de pedidos e reduz erros de comunicação entre cliente e equipe de vendas.

\textbf{Fluxo Principal:}
\begin{enumerate}
  \item O cliente acessa o site ou aplicativo de pedidos online.
  \item O cliente navega pelos produtos disponíveis e seleciona os itens desejados.
  \item O sistema registra o pedido do cliente, armazenando as informações dos produtos selecionados.
  \item O sistema verifica a disponibilidade dos produtos em estoque.
  \item Se os produtos estiverem disponíveis, o sistema gera uma confirmação de pedido e envia ao cliente.
  \item Se algum produto estiver indisponível, o sistema informa ao cliente e fornece opções alternativas, se disponíveis.
  \item O cliente confirma o pedido e realiza o pagamento, se necessário.
\end{enumerate}


\section*{ Geração Automática de Relatórios de Vendas}

\textbf{Descrição:} O sistema é configurado para gerar automaticamente relatórios de vendas com base em critérios como período, produto, cliente, entre outros. Os relatórios podem ser programados para serem gerados em intervalos regulares ou sob demanda, fornecendo insights sobre o desempenho das vendas de forma rápida e eficiente.

\textbf{Fluxo Principal:}
\begin{enumerate}
  \item O usuário do sistema seleciona a opção de geração de relatórios de vendas.
  \item O usuário define os critérios do relatório, como período, produto, cliente, entre outros.
  \item O sistema busca os dados relevantes do banco de dados.
  \item O sistema gera o relatório com base nos dados selecionados.
  \item O sistema exibe o relatório para o usuário.
\end{enumerate}


\section*{ Acompanhamento Automatizado de Pedidos}

\textbf{Descrição:} O sistema implementa um sistema de rastreamento de pedidos em tempo real, permitindo que os clientes e a equipe de atendimento ao cliente acompanhem o status dos pedidos desde o processamento até a entrega final. Isso reduz a necessidade de comunicação manual constante e proporciona uma experiência melhor para os clientes.

\textbf{Fluxo Principal:}
\begin{enumerate}
  \item O cliente acessa o sistema de rastreamento de pedidos.
  \item O sistema solicita ao cliente o número do pedido ou informações de identificação.
  \item O sistema busca as informações do pedido no banco de dados.
  \item O sistema exibe o status atual do pedido, como processamento, embalagem, transporte, etc.
  \item O cliente pode acompanhar o progresso do pedido em tempo real.
\end{enumerate}

\chapter{Relacionamentos e Requisitos Não Funcionais}\label{rel_e_requi_NF}

\section{Relacionamentos entre casos de uso}\label{rel_entre_casos}

\begin{enumerate}
	\item O caso de uso "Registro Automático de Entrada de Produtos" pode incluir o caso de uso "Cálculo Automático de Estoque", pois a entrada de produtos afeta o cálculo do saldo de estoque.
	\item O caso de uso "Pedidos Online Automatizados" pode incluir o caso de uso "Registro Automático de Entrada de Produtos", uma vez que o registro de entrada de produtos ocorre quando um pedido é feito e produtos são adicionados ao estoque.
	\item O caso de uso "Pedidos Online Automatizados" também pode incluir o caso de uso "Acompanhamento Automatizado de Pedidos", pois, após fazer um pedido, os clientes podem acompanhar o status do pedido através do sistema de rastreamento.
	\item O caso de uso "Geração Automática de Relatórios de Vendas" não apresenta relacionamentos diretos com outros casos de uso neste conjunto.
\end{enumerate}

\section{Requisitos Não Funcionais }\label{req_n_func}

\begin{itemize}
	\item Desempenho: O sistema deve ser capaz de lidar com um grande volume de transações, processando-as de forma eficiente e mantendo tempos de resposta rápidos.
	\item Segurança: O sistema deve garantir a proteção dos dados dos clientes, produtos e transações, implementando medidas de segurança, como criptografia e autenticação adequadas.
	\item Usabilidade: O sistema deve ser intuitivo e fácil de usar, tanto para os funcionários responsáveis pelo controle de estoque quanto para os clientes que acessam o site ou aplicativo de pedidos online. Deve ser fornecida uma interface amigável e instruções claras para facilitar a interação com o sistema.
	\item Confiabilidade: O sistema deve ser altamente confiável, garantindo que as transações sejam registradas corretamente, que os dados estejam sempre disponíveis e que não ocorram perdas de informações importantes.
	\item Escalabilidade: O sistema deve ser capaz de lidar com o crescimento do volume de dados e transações ao longo do tempo, sem comprometer seu desempenho e funcionalidade.
	\item Manutenibilidade: O sistema deve ser desenvolvido de forma modular e organizada, facilitando a manutenção, atualização e expansão futuras.
	\item Integração: O sistema deve ser capaz de se integrar a outros sistemas existentes na empresa, como bancos de dados, sistemas de pagamento e sistemas de logística, garantindo a troca de informações de forma eficiente e precisa.
\end{itemize}


% ----------------------------------------------------------
% PARTE
% ----------------------------------------------------------
\part{Referenciais teóricos}
% ----------------------------------------------------------

% ---
% Capitulo de revisão de literatura
% ---
\chapter{Lorem ipsum dolor sit amet}
% ---

% ---
\section{Aliquam vestibulum fringilla lorem}
% ---

\lipsum[1]

\lipsum[2-3]

% ----------------------------------------------------------
% PARTE
% ----------------------------------------------------------
\part{Resultados}
% ----------------------------------------------------------

% ---
% primeiro capitulo de Resultados
% ---
\chapter{Lectus lobortis condimentum}
% ---

% ---
\section{Vestibulum ante ipsum primis in faucibus orci luctus et ultrices
posuere cubilia Curae}
% ---

\lipsum[21-22]

% ---
% segundo capitulo de Resultados
% ---
\chapter{Nam sed tellus sit amet lectus urna ullamcorper tristique interdum
elementum}
% ---

% ---
\section{Pellentesque sit amet pede ac sem eleifend consectetuer}
% ---

\lipsum[24]

% ----------------------------------------------------------
% Finaliza a parte no bookmark do PDF
% para que se inicie o bookmark na raiz
% e adiciona espaço de parte no Sumário
% ----------------------------------------------------------
\phantompart

% ---
% Conclusão
% ---
\chapter{Conclusão}
% ---

\lipsum[31-33]

% ----------------------------------------------------------
% ELEMENTOS PÓS-TEXTUAIS
% ----------------------------------------------------------
\postextual
% ----------------------------------------------------------

% ----------------------------------------------------------
% Referências bibliográficas
% ----------------------------------------------------------
\bibliography{abntex2-modelo-references}

% ----------------------------------------------------------
% Glossário
% ----------------------------------------------------------
%
% Consulte o manual da classe abntex2 para orientações sobre o glossário.
%
%\glossary

% ----------------------------------------------------------
% Apêndices
% ----------------------------------------------------------

% ---
% Inicia os apêndices
% ---
\begin{apendicesenv}

% Imprime uma página indicando o início dos apêndices
\partapendices

% ----------------------------------------------------------
\chapter{Quisque libero justo}
% ----------------------------------------------------------

\lipsum[50]

% ----------------------------------------------------------
\chapter{Nullam elementum urna vel imperdiet sodales elit ipsum pharetra ligula
ac pretium ante justo a nulla curabitur tristique arcu eu metus}
% ----------------------------------------------------------
\lipsum[55-57]

\end{apendicesenv}
% ---


% ----------------------------------------------------------
% Anexos
% ----------------------------------------------------------

% ---
% Inicia os anexos
% ---
\begin{anexosenv}

% Imprime uma página indicando o início dos anexos
\partanexos

% ---
\chapter{Morbi ultrices rutrum lorem.}
% ---
\lipsum[30]

% ---
\chapter{Cras non urna sed feugiat cum sociis natoque penatibus et magnis dis
parturient montes nascetur ridiculus mus}
% ---

\lipsum[31]

% ---
\chapter{Fusce facilisis lacinia dui}
% ---

\lipsum[32]

\end{anexosenv}

%---------------------------------------------------------------------
% INDICE REMISSIVO
%---------------------------------------------------------------------
\phantompart
\printindex
%---------------------------------------------------------------------

\end{document}
